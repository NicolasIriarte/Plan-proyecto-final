\section{15. Procesos de cierre}
\label{sec:cierre}

Al finalizar el proyecto se realizará una reunión final de evaluación del proyecto que contemplará
las siguientes actividades:

\begin{itemize}
\item Pautas de trabajo que se seguirán para analizar si se respetó el plan de proyecto original:

  \begin{itemize}
  \item  Nicolás Iriarte comparará junto al director, los tiempos establecidos en el plan de proyecto con los tiempos que se obtuvieron en la realización del proyecto.
  \item Nicolás Iriarte junto al director verificarán que se hayan cumplido la totalidad de los requerimientos solicitados por el cliente.
  \end{itemize}

\item Identificación de las técnicas y procedimientos  ́utiles e in ́utiles que se emplearon, y los problemas que surgieron y cómo se solucionaron:

  \begin{itemize}
  \item Nicolás Iriarte dejar ́a registro de todos los procedimientos utilizados a lo largo del proyecto, e identificará cuáles fueron los más  ́utiles y eficientes, así como también, los que no lo fueron.
  \item Nicolás Iriarte dejará registro de todos los problemas que hayan surgido durante la realización del proyecto y cómo se solucionaron.
  \end{itemize}

\item Indicar quién organizará el acto de agradecimiento a todos los interesados, y en especial al equipo de trabajo y colaboradores:

  \begin{itemize}
  \item Nicolás Iriarte presentará la defensa ante el jurado evaluador, en donde agradecerá al director, por su colaboración, tiempo, ayuda y compromiso con el proyecto; al cliente, por su confianza y por ofrecer la oportunidad de poder desarrollar el proyecto; y al jurado allí presente por su tiempo y disposición.
  \end{itemize}

\end{itemize}










\section{14. Gestión de la calidad}
\label{sec:calidad}

\begin{itemize}

  %%%%%%%%%%%%%%%%%%%%%%%%%%%%%%%%%%%%%%%%%%%%%%%%%%%%%%%%%%%%%%%%%%%%%%%%
\item Req #1: el emulador deberá ejecutar los mismos binarios que se utilizan en el hardware real.

  \begin{itemize}
  \item Verificación: tomar un software conocido que haya sido ejecutado en el hardware real.
  \item Validación: chequear que el emulador ejecute dicho software y finalize sin errores.
  \end{itemize}

  %%%%%%%%%%%%%%%%%%%%%%%%%%%%%%%%%%%%%%%%%%%%%%%%%%%%%%%%%%%%%%%%%%%%%%%%
\item Req #2: el sistema debe ser compatible con el sistema operativo Linux.

  \begin{itemize}
  \item Verificación: no aplica.
  \item Validación: sobre un entorno Linux chequear que el emulador se ejecute.
  \end{itemize}

  %%%%%%%%%%%%%%%%%%%%%%%%%%%%%%%%%%%%%%%%%%%%%%%%%%%%%%%%%%%%%%%%%%%%%%%%
\item Req #3: el emulador deberá poder ejecutar correctamente parte del set de instrucciones del procesador real.

  \begin{itemize}
  \item Verificación: tomar algún binario que cuya ejecución deje la memoria en un estado conocido.
  \item Validación: se la ejecución deje la memoria emulada en dicho estado.
  \end{itemize}

  %%%%%%%%%%%%%%%%%%%%%%%%%%%%%%%%%%%%%%%%%%%%%%%%%%%%%%%%%%%%%%%%%%%%%%%%
\item Req #4: se deberán desarrollar tests unitarios que verifiquen el set de instrucciones desarrollado.

  \begin{itemize}
  \item Verificación: no aplica.
  \item Validación: se podrán compilar y ejecutar dichos tests para verificar su funcionamiento.
  \end{itemize}

  %%%%%%%%%%%%%%%%%%%%%%%%%%%%%%%%%%%%%%%%%%%%%%%%%%%%%%%%%%%%%%%%%%%%%%%%
\item Req #5: se deberá desarrollar un ambiente de automatización de pruebas en Gitlab CI.

  \begin{itemize}
  \item Verificación: no aplica.
  \item Validación: se entregarán reporte de tests ejecutados por el CI.
  \end{itemize}

  %%%%%%%%%%%%%%%%%%%%%%%%%%%%%%%%%%%%%%%%%%%%%%%%%%%%%%%%%%%%%%%%%%%%%%%%
\item Req #6: el software podrá utilizarse como biblioteca compartida.

  \begin{itemize}
  \item Verificación: se entregarán los \textit{headers} y librería compartida para poder extender la funcionalidad y/o integrarla en otros ambientes.
  \item Validación: inspeccionar el entregable.
  \end{itemize}

  %%%%%%%%%%%%%%%%%%%%%%%%%%%%%%%%%%%%%%%%%%%%%%%%%%%%%%%%%%%%%%%%%%%%%%%%
\item Req #7: la API expuesta deberá estar documentada con Doxygen.

  \begin{itemize}
  \item Verificación: no aplica.
  \item Validación: el cliente tendrá acceso a dicha documentación.
  \end{itemize}

  %%%%%%%%%%%%%%%%%%%%%%%%%%%%%%%%%%%%%%%%%%%%%%%%%%%%%%%%%%%%%%%%%%%%%%%%
\item Req #8: se realizará un manual de usuario que describa los funcionamientos clave del software.

  \begin{itemize}
  \item Verificación: no aplica.
  \item Validación: inspección del manual.
  \end{itemize}

  %%%%%%%%%%%%%%%%%%%%%%%%%%%%%%%%%%%%%%%%%%%%%%%%%%%%%%%%%%%%%%%%%%%%%%%%
\item Req #10: la API expuesta deberá ser en C.

  \begin{itemize}
  \item Verificación: no aplica.
  \item Validación: el cliente tendrá acceso al código fuente de la API.
  \end{itemize}

\end{itemize}

%% Tener en cuenta que en este contexto se pueden mencionar simulaciones, cálculos, revisión de hojas de datos, consulta con expertos, mediciones, etc.

%% Las acciones de verificación suelen considerar al entregable como ``caja
%% blanca'', es decir se conoce en profundidad su funcionamiento interno.

%%En cambio, las acciones de validación suelen considerar al entregable como
%%``caja negra'', es decir, que no se conocen los detalles de su funcionamiento
%%interno.

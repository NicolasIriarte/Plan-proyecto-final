\section{6. Requerimientos}
\label{sec:6-requerimientos}

\begin{enumerate}
\item Requerimientos funcionales
  \begin{enumerate}
  \item El emulador deberá ejecutar los mismos binarios que se utilizan en el hardware real.
  \item El sistema debe ser compatible con el sistema operativo Linux.
  \item El emulador deberá poder ejecutar correctamente parte del set de instrucciones del procesador real.
  \item Se deberán desarrollar tests unitarios que verifiquen el set de instrucciones desarrollado.
  \item Se deberá desarrollar un ambiente de automatización de pruebas en Gitlab CI.
  \item El software podrá utilizarse como biblioteca compartida.
  \end{enumerate}

\item Requerimientos de documentación
  \begin{enumerate}
  \item La API expuesta deberá estar documentada con Doxygen.
  \item Se realizará un manual de usuario que describa los funcionamientos clave del software.
    %% from: https://lse-posgrados-files.fi.uba.ar/tesis/LSE-FIUBA-Trabajo-Final-CEIA-Maria-Carina-Roldan-2022-Plan.pdf
  \item Se deberá redactar un informe de avance y la memoria técnica final del proyecto.
  \end{enumerate}

\item Requerimientos de desarrollo de software
  \begin{enumerate}
  \item El software deberá ser escrito en C++.
  \item La API expuesta deberá ser en C.
  \item El software debe mantenerse bajo control de versiones en Gitlab.
  \end{enumerate}
\end{enumerate}



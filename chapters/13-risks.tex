
\section{13. Gestión de riesgos}
\label{sec:riesgos}

\begin{enumerate}[label=\alph*)] % Lowercase alphabetic enumeration
\item Identificación de los riesgos y estimación de sus consecuencias:

  A continuación se detallan cinco posibles riesgos inherentes al proyecto. Los mismos son evaluados según su grado de severidad y su probabilidad de ocurrrencia tomando valores del 1 (bajo) al 10 (alto).

  %%%%%%%%%%%%%%%%%%%%%%%%%%%%%%%%%%%%%%%%%%%%%%%%%%%%%%%%%%%%%%%%%%%%%
  \textbf{Riesgo 1}: no finalizar el proyecto en el plazo de tiempo establecido.

  \begin{itemize}
  \item Severidad (S): 10. Severidad alta porque no se cumpliría con la fecha de entrega límite establecida.
  \item Ocurrecia (O): 6. Es probable ya que la totalidad del proyecto se realizará fuera del horario laboral.
  \end{itemize}

  %%%%%%%%%%%%%%%%%%%%%%%%%%%%%%%%%%%%%%%%%%%%%%%%%%%%%%%%%%%%%%%%%%%%%
  \textbf{Riesgo 2}: dificultad en la capacitación en el \textit{framework} LLVM.


  \begin{itemize}
  \item Severidad (S): 8. Severidad media/alta porque se planea tomar varias funcionalidad de dicho \textit{framework}, que de otra manera implicarían un esfuerzo mucho mayor.
  \item Ocurrecia (O): 5. Es probable que suceda ya que es un proyecto de alta complejidad, pero al mismo tiempo tiene mucha documentación asociada.
  \end{itemize}

  %%%%%%%%%%%%%%%%%%%%%%%%%%%%%%%%%%%%%%%%%%%%%%%%%%%%%%%%%%%%%%%%%%%%%
  \textbf{Riesgo 3}: bajo desempeño del emulador, con ejecuciones que no alcancen al tiempo real.

  \begin{itemize}
  \item Severidad (S): 3. Baja severidad, ya que se plantea un prototipo sin funcionanlidad completa, que luego podría ser optimizado para cumplir los requerimientos de performance.
  \item Ocurrecia (O): 6. Es probable que suceda ya que se trata de un \textit{software} de alta complejidad computacional.
  \end{itemize}

  %%%%%%%%%%%%%%%%%%%%%%%%%%%%%%%%%%%%%%%%%%%%%%%%%%%%%%%%%%%%%%%%%%%%%
  \textbf{Riesgo 4}: dificultad en la comprensión de documentos, tanto de \textit{datasheets} como de manuales de arquitectura del procesador.


  \begin{itemize}
  \item Severidad (S): 10. Alta severidad, ya que sin entender dichos documentos no es posible desarrollar software que cumpla el correcto funcionamiento del procesador.
  \item Ocurrecia (O): 8. Se estima una alta probabilidad de ocurrencia debido a que son documentos sumamente extensos y de gran detalle técnico.
  \end{itemize}

  %%%%%%%%%%%%%%%%%%%%%%%%%%%%%%%%%%%%%%%%%%%%%%%%%%%%%%%%%%%%%%%%%%%%%
  \textbf{Riesgo 5}: dificultad para localizar y depurar errores de desarrollo del \textit{software}.


  \begin{itemize}
  \item Severidad (S): 4. Media severidad, ya que al estár internalizado con el código fuente, todas las fuentes de errores son facilmente identificables.
  \item Ocurrecia (O): 8. Media/alta ocurrencia, ya que es normal cometer errores en la programación de código.
  \end{itemize}

  %%%%%%%%%%%%%%%%%%%%%%%%%%%%%%%%%%%%%%%%%%%%%%%%%%%%%%%%%%%%%%%%%%%%%

\item Tabla de gestión de riesgos: (El RPN se calcula como RPN = $\text{S} \times \text{O}$)


  \begin{table}[htpb]
    \centering
    \begin{tabularx}{\linewidth}{@{}|X|c|c|c|c|c|c|@{}}
      \hline
      \rowcolor[HTML]{C0C0C0}
      Riesgo & S & O & RPN & S* & O* & RPN* \\ \hline
      1. No finalizar el proyecto en el plazo de tiempo establecido. & 10  & 6  &  60   &  10  &  4  &   40   \\ \hline
      2. Dificultad en la capacitación en el \textit{framework} LLVM. & 8  & 5  &  40   &  -  &  -  &   -   \\ \hline
      3. Bajo desempeño del emulador, con ejecuciones que no alcancen al tiempo real. &  3 & 6  &  18   &  -  &  -  &   -   \\ \hline
      4. Dificultad en la comprensión de documentos, tanto de \textit{datasheets} como de manuales de arquitectura del procesador. & 10  & 8  &  80   &  7  &  7  &  49    \\ \hline
      5. Dificultad para localizar y depurar errores de desarrollo del \textit{software}. &  4 & 8  &  32   &  -  & -  &   -   \\ \hline
    \end{tabularx}
  \end{table}

  Criterio adoptado: se tomarán medidas de mitigación en los riesgos cuyos números de RPN sean mayores a 50.

  Nota: los valores marcados con (*) en la tabla corresponden luego de haber aplicado la mitigación.

\item Plan de mitigación de los riesgos que orifginalmente excedían el RPN máximo establecido.

  \textbf{Riesgo 1}: se asignarán horas fijas semanales para garantizar que el proyecto se lleve a cabo en
  el tiempo estipulado.

  \begin{itemize}
  \item Severidad (S*): 10. Se mantiene la misma severidad que antes de la mitigación.
  \item Ocurrecia (O*): 4. Baja significativamente la probabilidad de ocurrencia de este riesgo ya
que se tratará de ser riguroso con el seguimiento y control del plan de proyecto estipulado.
  \end{itemize}

  \textbf{Riesgo 4}: se utilizará el modelo de referencia provisto por la empresa para despejar dudas sobre la documentación respecto al funcionamiento esperado.

  \begin{itemize}
  \item Severidad (S*): 7. Se disminuye la severidad, ya que la documentación deja de ser el único punto de referencia para establecer el correcto funcionamiento del procesador.
  \item Ocurrecia (O*): 7. Reduce la ocurrencia levemente porque se chequearán ambas fuentes de verdad.
  \end{itemize}

\end{enumerate}


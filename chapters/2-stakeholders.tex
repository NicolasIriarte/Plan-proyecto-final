\section{2. Identificación y análisis de los interesados}
\label{sec:interesados}

\begin{table}[ht]
%\caption{Identificación de los interesados}
%\label{tab:interesados}
\begin{tabularx}{\linewidth}{@{}|l|X|X|l|@{}}
\hline
\rowcolor[HTML]{C0C0C0}
Rol           & Nombre y Apellido & Organización 	 & Puesto 	\\ \hline
Cliente       & Lic. \clientename      &\empclientename &   Especialista Tecnológico     	\\ \hline
Responsable   & \authorname       & FIUBA        	 & Alumno 	\\ \hline
Orientador    & Esp. Lic. \supname	        & \pertesupname  & Director Trabajo final \\ \hline
Usuario final    & Departamento de Embebidos y Sistemas Críticos & \pertesupname  & - \\ \hline
\end{tabularx}
\end{table}

\begin{itemize}
\item Cliente: Lic. Matías Pinedo, del Departamento de Embebidos y Sistemas Críticos de la empresa INVAP S.E., quien aporta su visión y experiencia necesaria para desarrollar el proyecto.
\item Orientador: Es el director del trabajo final, quien aportará sus conocimientos técnicos y experiencia para la realización del proyecto.
\item Usuario final: será el Departamento de Embebidos y Sistemas Críticos de la empresa INVAP S.E.
\end{itemize}

%% \end{consigna} % este comando se debe borrar para la entrega, junto con la contraparte \begin{consigna}{red}


\section{7. Historias de usuarios (\textit{Product backlog})}
\label{sec:7-historias-de-usuarios-product-backlog}


Roles:

\begin{itemize}
\item Desarrollador de modelos simulados: quien se encarga de desarrollar dispositivos de manera simulada. Tales como podrían ser memorias, FPGAs y/o periféricos.
\item Desarrollador de software de vuelo: quien se encarga de desarrollar un driver o software que usa interfaces de bajo nivel. Durante el diseño e implementación, utiliza este sistema para interactuar con el hardware.
\item Ingeniero de pruebas: quien se encarga de automatizar los ensayos de calificación de hardware y/o de componentes que lo emulen.
\end{itemize}

Story Points:

\begin{itemize}
\item Se analizarán las historias según dificultad, complejidad e incertidumbre, tomando valores de Fibonacci usando el siguiente criterio:
  \begin{enumerate}
  \item Dificultad: cantidad de trabajo a realizar. Representado con la letra ``\textbf{D}''.

	  \begin{itemize}
	  \item Bajo: 1
	  \item Medio: 3
	  \item Alto: 13
	  \end{itemize}

  \item Complejidad: complejidad de trabajo a realizar. Representado con la letra ``\textbf{C}''.

	  \begin{itemize}
	  \item Bajo: 1
	  \item Medio: 5
	  \item Alto: 13
	  \end{itemize}

  \item Incertidumbre: Riesgo del trabajo a realizar. Representado con la letra ``\textbf{I}''.

	  \begin{itemize}
	  \item Bajo: 1
	  \item Medio: 3
	  \item Alto: 5
	  \end{itemize}

  \end{enumerate}

\end{itemize}

Historias de Usuarios:

\begin{itemize}
\item Como desarrollador de modelos simulados quiero poder comunicarme con el software de vuelo. D: 3, C: 5, I: 5 Total = 13p.

\item Como desarrollador de modelos simulados quiero poder ejecutar el software de vuelo como una librería. D: 3, C: 1, I: 1. Total = 5p.

\item Como desarrollador de software de vuelo quiero poder estimular los registros, memorias y hardware para el desarrollo de controladores (\textit{drivers}). D: 13, C: 13, I: 5. Total = 34p.

\item Como desarrollador de software de vuelo quiero que mi software corra a tiempo real. D: 5, C: 5, I: 5. Total = 21p.

\item Como ingeniero de pruebas quiero que mis pruebas puedan ser ejecutadas automáticamente en un entonces de desarrollo continuo (por ejemplo GitLab CI). D: 3, C: 2, I: 2. Total = 8p.
\end{itemize}
